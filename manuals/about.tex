\section{关于模板}

\subsection{求助与反馈}

请follow我们最新的进度\footnote{\url{https://github.com/Tr0py/NKU-thesis-template-2020}}, 
欢迎加入毕业论文模板维护的工作中。如果遇到使用问题,你可以用以下方式求助:
\begin{itemize}
  \item 在 \href{https://github.com/Tr0py/NKU-thesis-template-2020/discussions}{GitHub讨论区} 询问你不懂的地方。
  \item 在 \href{https://github.com/Tr0py/NKU-thesis-template-2020/issues}{GitHub Issue} 中提出你遇到的待修复的问题。
  \item 如果不会使用GitHub,你可以在QQ群中提问:群号469868290。
\end{itemize}

\subsection{贡献者}

感谢历届的贡献者们:
\begin{itemize}
  \item 赵子懿,20届本科毕业生,\url{troppingz@gmail.com},\href{https://tr0py.github.io/}{主页}
  \item 潘宇,20届本科毕业生,\url{panyu@nbjl.nankai.edu.cn},\href{https://nbjl.nankai.edu.cn/2019/0513/c19244a264683/page.htm}{主页}
  \item 黎鸿儒,22届本科毕业生,\url{lihongru@mail.nankai.edu.cn}
  \item 林雪,22届本科毕业生,\url{ylsnowy@gmail.com}
  \item 费迪,22届本科毕业生,\url{fd@nbjl.nankai.edu.cn}
  \item 安祺,23届本科毕业生,\url{aqnin@outlook.com}
\end{itemize}

\subsection{发展历史}

\subsubsection{发源}

\LaTeX 由于其专业的排版效果,擅长处理复杂的数学公式,免费等优势而得到了广泛的使用,成为许多西文期刊接收投稿的主流格式;而\LaTeX 环境的安装,编辑器的选择以及编译\LaTeX 文件的设置等增加了使用者的使用困难,并且在此过程中出现的问题有难有易,种类繁多,不易找到解决方案。Overleaf作为一档在线\LaTeX 编辑网站,使用简单,支持在线和跨平台编译修改毕业论文,并且支持多人协作以方便老师和项目组其他人修改。因此,我们参考南开大学08级本科生卓小杨依据《南开大学本科生毕业论文(设计)管理规定(2011年 修订)》修改制成的本科生毕业论文模板,将南开大学孙文昌老师制作的南开大学研究生论文\LaTeX 模板移植到了Overleaf上,借鉴未知人士对其的修改,针对该模板的若干问题和版本旧的不足进行了修改及更新。

本模板参考《南开大学本科生学位论文排版规范(试行)(2016 修正)版本 1.2.21》修改而成,更新了封面格式、添加了所需字体、更新参考文献到最新的bibtex标准。作者希望能给使用者写作论文带来方便。然而,作者不保证本模板完全符合学校要求,也不对由此带来的风险和损失承担任何责任。如果您不接受这项条款,请不要使用本模板。

\hfill 16级本科生\quad 赵子懿\quad 潘宇

\subsubsection{迭代}

在2023年5月,本模板经历了一次重构。根据 \href{http://jwc.nankai.edu.cn/2022/1124/c24a497818/page.htm}{2022年11月版《南开大学本科毕业论文(设计)指导手册》} 的格式要求重新进行了实现,不再兼容原模板使用接口。

\hfill 19级本科生\quad 安祺