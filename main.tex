% -*- coding: utf-8 -*-
% !TEX program = xelatex
% !TEX encoding = utf-8

\documentclass{nkuthesis}

% 参考文献
\usepackage[maxnames=3,minnames=3,sorting=none,style=nkthesis]{biblatex}
\addbibresource{nkthesis.bib}

% 数学环境
\usepackage{amssymb}
\usepackage{amsmath}
\usepackage{newtxmath}
\newtheorem{Theorem}{\hspace{2em}定理}[section]
\newtheorem{Lemma}[Theorem]{\hspace{2em}引理}
\newtheorem{Corollary}[Theorem]{\hspace{2em}推论}
\newtheorem{Proposition}[Theorem]{\hspace{2em}命题}
\newtheorem{Definition}[Theorem]{\hspace{2em}定义}
\newtheorem{Example}[Theorem]{\hspace{2em}例}

% 列表格式
\usepackage{enumitem}
\setlist{nosep}

% 编号格式
\usepackage{chngcntr}
\counterwithin{table}{section}
\counterwithin{figure}{section}
\counterwithin{equation}{section}

% 页面风格
\usepackage{fancyhdr}
\pagestyle{fancy}

% 其他
\usepackage{booktabs} % 三线表
\usepackage{graphicx} % 图片导入
\hypersetup{ % 超链接样式
  colorlinks=true,    
  pdfborder=0 0 1,
  citecolor=black,
  linkcolor=black,
}
\usepackage{hologo} % logo
\usepackage{xurl}

% 引用编号格式
\labelformat{section}{第\zhnumber{#1}章}
\labelformat{subsection}{第\zhnumber{#1}节}
\labelformat{figure}{图~#1\space}

% 文档元数据
\title{南开大学本科生毕业论文模板\\ v3.0}
\title*{Released in May, 2023}
\studentid{1234567}
\author{XXX}
\grade{2019 级}
\major{计算机科学与技术}
\department{计算机科学与技术}
\college{计算机学院}
\adviser{XXX\quad 教授 \\ 罗翔\quad 教授 \\(张三大学)}
\date{\today}
\keywords{南开大学;毕业论文;模板}
\keywords*{dark energy model; dynamic analysis}

\begin{document}

\pagenumbering{Roman}

\maketitle
\begin{titlepage}
  \declaration
\end{titlepage}

\begin{abstract}
此模板最初按照南开大学计算机/网络空间安全学院本科生毕业设计要求制作整理。我们希望此模板能够帮助更多的同学。本模板会根据同学们的需求进行更新。最新版本请关注本模版主页:\href{https://tr0py.github.io/NKU-thesis-template-2020/}{NKThesis: 南开大学本科生毕业论文模板}。

在2023年5月,本模板经历了一次较大的重构,根据\href{http://jwc.nankai.edu.cn/2022/1124/c24a497818/page.htm}{2022年11月版《南开大学本科毕业论文(设计)指导手册》}重新进行了实现,原有接口不再兼容。
  
如果遇到使用问题,你可以用以下方式求助:
\begin{itemize}
  \item 在\href{https://github.com/Tr0py/NKU-thesis-template-2020/discussions}{GitHub讨论区}询问你不懂的地方。
  \item 在\href{https://github.com/Tr0py/NKU-thesis-template-2020/issues}{GitHub Issue}中提出你遇到的待修复的问题。
  \item 如果不会使用GitHub,你可以在QQ群中提问:群号469868290。
\end{itemize}

感谢历届的贡献者们:
\begin{itemize}
    \item 安祺,23届本科毕业生,\url{aqnin@outlook.com}
    \item 费迪,22届本科毕业生,\url{fd@nbjl.nankai.edu.cn}
    \item 林雪,22届本科毕业生,\url{ylsnowy@gmail.com}
    \item 黎鸿儒,22届本科毕业生,\url{lihongru@mail.nankai.edu.cn}
    \item 潘宇,20届本科毕业生,\url{panyu@nbjl.nankai.edu.cn},\href{https://nbjl.nankai.edu.cn/2019/0513/c19244a264683/page.htm}{主页}
    \item 赵子懿,20届本科毕业生,\url{troppingz@gmail.com},\href{https://tr0py.github.io/}{主页}
\end{itemize}
\end{abstract}
  
\begin{abstract*}
Since 1998, two independent supernova research groups have discovered that the universe is accelerating, and the dark energy has become a hot topic in cosmology. 
\end{abstract*}

\tableofcontents

\pagenumbering{arabic} 

\section{介绍} \label{chpt:A}

\subsection{维护中}

请follow我们最新的进度 \url{https://github.com/Tr0py/NKU-thesis-template-2020}, 
以获取最新的模板bug修复及更新。如有问题请在GitHub留言,我们会尽快回复。
欢迎加入毕业论文模板维护的工作中。

\subsection{免责声明}

本模板本模板参考\href{http://jwc.nankai.edu.cn/2022/1124/c24a497818/page.htm}
{2022年11月版《南开大学本科毕业论文(设计)指导手册》}编写。

您自愿使用这个模板。
提供本模板的目的是为了给您的论文写作带来方便,然而,
作者不保证这个模板完全符合学校的要求,也不对由此产生的任何后果负责。
如果您不同意这些条款,请不要使用这个模板。

\include{manual}
\appendix
单标题附录\footnote{脚注}

\appendices
\section{标题}
\section{标题}
\section{标题}
\section{标题}
\section{标题}\section{标题}\section{标题}\section{标题}\section{标题}

\begin{reference}
\printbibliography[heading=none]
\end{reference}

\begin{acknowledgement}
  感谢使用本模板。
\end{acknowledgement}

\end{document}
