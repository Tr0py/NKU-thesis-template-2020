% -*- coding: utf-8 -*-
\documentclass[12pt,openright]{book}

\usepackage{ifxetex}
\ifxetex
  \usepackage[bookmarksnumbered]{hyperref}
\else
  \usepackage[unicode,bookmarksnumbered]{hyperref}
\fi
% 上下2.54cm,左右3.17cm,页眉1.5cm,页脚1.75cm
\usepackage{geometry}
%分学院风格的模板。已实现的:Chem/化院,Bio/生科院
\def\college{Bio}
\usepackage[emptydoublepage,\college]{NKThesis}   % 中文
%\usepackage[emptydoublepage,English]{NKThesis} % 英文
\usepackage{amssymb}
%   根据需要选择 biblatex 宏包选项.
%\usepackage[maxnames=3,minnames=3,sorting=none]{biblatex}
\usepackage[maxnames=3,minnames=3,sorting=none,  style = nkthesis]{biblatex}
% \usepackage{cite}
\hypersetup{colorlinks=true,
            pdfborder=0 0 1,
            citecolor=black,
            linkcolor=black}
%\usepackage{tikz}

\usepackage{amsmath}
\usepackage{booktabs}
\usepackage{graphicx}
\usepackage{multirow}
\usepackage{pgfplots}
\pgfplotsset{compat=1.13}
\usepackage[labelsep=space]{caption}
\usepackage[utf8]{inputenc}
\usepackage{graphicx} 
\usepackage{subfigure}
\usepackage{listings}
\usepackage{pgfplots}
\usepackage{tikz}
\usepackage{setspace}
\usepackage{color,xcolor}
\usepackage{amsmath}
\usepackage{amssymb}
\usepackage{algorithm}
\usepackage{algpseudocode}
\usepackage{cases}
\usepackage{multirow}
\usepackage{tabularx}
\definecolor{corange}{HTML}{FF9666}
\definecolor{lblue}{HTML}{97FFFF}
\definecolor{cblue}{HTML}{AB82FF}
\definecolor{cred}{HTML}{FF99A1}
\definecolor{cgreen}{HTML}{7ACCBE}
\definecolor{red1}{RGB}{254,67,101}
\definecolor{red2}{RGB}{252,157,154}
\definecolor{red3}{RGB}{249,205,173}
\definecolor{grey1}{RGB}{200,200,169}
\definecolor{grey2}{RGB}{131,175,155}
\definecolor{color1}{HTML}{0099CC}
\definecolor{color2}{HTML}{F9B46F}
\definecolor{cblue1}{HTML}{0074D9}
\definecolor{lblue1}{HTML}{99CCFF}
\definecolor{corange1}{HTML}{FFDDCE}
\definecolor{corange2}{HTML}{FF9900}
\definecolor{corange3}{HTML}{FF9966}
\definecolor{corange4}{HTML}{FF6600}
\definecolor{cred2}{HTML}{FF6666}
\usetikzlibrary{patterns}
\addbibresource{nkthesis.bib}
\DeclareBibliographyCategory{cited}
\AtEveryCitekey{\addtocategory{cited}{\thefield{entrykey}}}
\newtheorem{Theorem}{\hskip 2em 定理}[chapter]
\newtheorem{Lemma}[Theorem]{\hskip 2em 引理}
\newtheorem{Corollary}[Theorem]{\hskip 2em 推论}
\newtheorem{Proposition}[Theorem]{\hskip 2em 命题}
\newtheorem{Definition}[Theorem]{\hskip 2em 定义}
\newtheorem{Example}[Theorem]{\hskip 2em 例}
\newcommand{\upcite}[1]{\textsuperscript{\textsuperscript{\cite{#1}}}}
\renewcommand{\algorithmicrequire}{\textbf{输入:}}
\renewcommand{\algorithmicensure}{\textbf{输出:}}

\ifdefstring{\college}{Chem}{
  \defbibheading{bibliography}[\bibname]{\reference}
  }{}
\renewcommand*{\songti}{\CJKfamily{zhsong}} %宋体加粗
\algnewcommand{\LeftComment}[1]{\Statex \(\triangleright\) #1}
\floatname{algorithm}{算法}
\begin{document}
% \captionsetup{font={\small}}

%  设置基本信息
%  注意:  逗号`,'是项目分隔符. 如果某一项的值出现逗号, 应放在花括号内, 如 {,}
%

\NKTsetup{%
  论文题目(中文) = 南开大学本科生毕业论文模板 V2.0,
  %论文题目(中文)(第二行) =第二行中文题目,填无则不显示本行,
  论文题目(中文)(第二行) = 修订版,
  论文题目(英文) =  {Released in Jan, 2023},
  论文题目(英文)(第二行) = abcd,
  % 论文题目(英文)(第二行) =无,
  学号           = 191xxxx,
  姓名          = 张三,
  年级          = 2019,
  级            = 级,
  专业           = 计算机科学与技术,
  系别          = 计算机科学与技术,
  学院          = 计算机学院,
  指导教师       = 李四\quad 教授,
  % 如果有校外导师则填写校外导师。没有填无
  校外导师      =无,
  %校外导师     = 罗翔\quad 教授(张三大学),
  论文完成时间   = 2023,
  年 = 年,
  月份 = 5 月,
  月 =,
  空后缀 = ,
  }

\ifdefstring{\college}{Chem}{\NKTdeclaration}{}
% -*- coding: utf-8 -*-


\begin{zhaiyao}
\begin{spacing}{1.5}
{


此模板按照南开大学计算机/网络空间安全学院2020年本科生毕业设计要求制作整理。我们希望此模板能够帮助更多的同学。我们希望不断改此模板,如有问题欢迎联系我们,也欢迎加入我们共同整理、更新。

你可以通过GitHub的Issue功能给我们提建议,如果我们没有及时回复,也可以直接联系整理者。

GitHub repo: \url{https://github.com/Tr0py/NKU-thesis-template-2020}

QQ群:469868290

整理者(按照姓氏拼音排序)
\begin{itemize}
    \item 潘宇,20届本科毕业生,\url{panyu@nbjl.nankai.edu.cn},主页\url{https://nbjl.nankai.edu.cn/2019/0513/c19244a264683/page.htm}
    \item 赵子懿,20届本科毕业生,\url{troppingz@gmail.com},主页\url{https://tr0py.github.io/}
\end{itemize}
}

\end{spacing}
\end{zhaiyao}




\begin{guanjianci}
南开大学,毕业论文,模板
\end{guanjianci}



\begin{abstract}
\begin{spacing}{1.5}
Since 1998, two independent supernova research groups have discovered that the universe is accelerating, and the dark energy has become a hot topic in cosmology. 

\end{spacing}
\end{abstract}


\begin{keywords}
dark energy model; dynamic analysis
\end{keywords} 
\tableofcontents
\begin{spacing}{1.5}
\include{manual}
% -*- coding: utf-8 -*-
% \chapter{参考文献}
\zihaowu
%\renewcommand{\bibname}{参考文献}
%\def\bibrangedash{ $\sim$ }
%\printbibliography
\def\bibrangedash{ $\sim$ }
\printbibliography [ category = cited]
\ifdefstring{\college}{Chem}{
  % -*- coding: utf-8 -*-

\begin{zhixie}
感谢使用本模板。
\end{zhixie}
  \end{spacing}
  }{
\ifdefstring{\college}{Bio}{
  % -*- coding: utf-8 -*-

\begin{zhixie}
感谢使用本模板。
\end{zhixie}
  \end{spacing}
  }{
  \end{spacing}
  % -*- coding: utf-8 -*-

\begin{zhixie}
感谢使用本模板。
\end{zhixie}
  % -*- coding: utf-8 -*-


\chapter*{个人简历}

\noindent 基本信息:\\

 姓名:~张三

 性别:~男 

 出生日期:~19xx年xx月xx日

 通信地址:~XXXXXXXXXXX

 电  话:xxxxxxxxxxx

 E-mail:~nb@nankai.edu.cn \\ \\
\\
教育背景:\\  

2016.09-2020.07 \quad 南开大学\quad 计算机学院\quad\quad 计算机科学与技术\quad\quad 学士 \\ 
%2012年9月-2016年7月\quad 南开大学\quad 某某学院\quad\quad\quad\quad 某某学\quad\quad 硕士 \\
\\
%硕士期间发表的学术论文:\\

}%Bio
}%Chem
\end{document}
